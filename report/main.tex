\documentclass{article}
\usepackage{graphicx} % Required for inserting images
\usepackage{listings} % Replacing minted with listings
\usepackage[margin=25mm]{geometry}
\usepackage[indent=0mm]{parskip}
\usepackage{braket}
\usepackage[math]{cellspace}
\usepackage{xcolor} % to access the named colour LightGray
\definecolor{LightGray}{gray}{0.9}
\usepackage[
backend=biber,
style=chem-acs,
sorting=none
]{biblatex}
\addbibresource{ref.bib}
\usepackage{lmodern}  % Latin Modern, a sharper font
\usepackage[T1]{fontenc}

\usepackage{listings}
\lstset{
    backgroundcolor=\color{LightGray},
    basicstyle=\ttfamily,
    breaklines=true,
    frame=single,
    language=Python,
    keywordstyle=\color{blue},
    commentstyle=\color{green!60!black},
    stringstyle=\color{red}
}

\title{Winter School Hückel Homework}
\author{Fernando Mata Rabadán}
\date{}
\makeindex
\begin{document}

\maketitle


\section{Theoretical background}
The focus of this homework is to write a program that is able to obtain the eigenvalues and eigenstates of the Hückel Hamiltonian in carbon chains and rings. The Hückel Hamiltonian allows calculating the energies of conjugated $\pi$ systems. 

The Hückel Hamiltonian is built based on two parameters: $\alpha$ and $\beta$. They are defined as:

\begin{equation} 
\alpha = -\frac{2}{2r^{2}}
\,\,\,\,\,\,\,\,\,\,\,\,\,\,
\beta = -\frac{1}{2r^{2}}
\end{equation} \label{eq:alpha_beta_equation}

Where $r$ is the distance of the electron with respect to the atom. The parameter $\alpha$ is the energy of an electron in a $2p$ orbital. The $\beta$ parameter represents the energy of stabilization of an electron due to the adjacent atoms. This definition shows that the value of $\alpha$ depends on the atom type, as the size of the $2p$ orbtal will vary with the atom. Also, the value of beta depends on the distance to the neigbouring atoms as it will affect the stabilization. 

Both $\alpha$ and $\beta$ are negative quantities, as they represent stabilization with respect to a free electron. The Hückel Hamiltonian matrix is built with $\alpha$ as the diagonal terms and $\beta$ in the off-diagonal terms of interacting atoms. A simple example of pentane atoms would be: 
\begin{equation}
    \hat{H}_{\textrm{Hückel}}  =
    \begin{pmatrix}
        \alpha & \beta & 0 & 0 & 0 \\
        \beta & \alpha & \beta & 0 & 0 \\
        0 & \beta & \alpha & \beta & 0 \\
        0 & 0 & \beta & \alpha & \beta \\
        0 & 0 & 0 & \beta & \alpha \\
    \end{pmatrix}
\end{equation}
    
Where the first atom is connected to the second, the second to the third and so on. By diagonalizing this Hamiltonian, the energies and wave functions of the system are found:
\begin{equation}
    \hat{H}_{\textrm{Hückel}}\ket{\Psi_i}   =  E_{i}\ket{\Psi_i}
\end{equation}

The Hückel Hamiltonian can also be applied to rings. In this case, this can be done by including the matrix elements that connect the first and last atoms. In the case of cyclopentane it would be: 
\begin{equation}
    \hat{H}_{\textrm{Hückel}}  =
    \begin{pmatrix}
        \alpha & \beta & 0 & 0 & \beta \\
        \beta & \alpha & \beta & 0 & 0 \\
        0 & \beta & \alpha & \beta & 0 \\
        0 & 0 & \beta & \alpha & \beta \\
        \beta & 0 & 0 & \beta & \alpha \\
    \end{pmatrix}
\end{equation}

As the quantity of interest is the relative energy difference between states, it is possible to set $\alpha=0$, as this does not affect the relative energy of eigenvalues. When there are different types of atoms, the value of $\alpha$ will not be the same, and thus this must be reflected in the Hamiltonian matrix. An example of a chain of 5 alternating atoms would be:
\begin{equation}
    \hat{H}_{\textrm{Hückel}}  =
    \begin{pmatrix}
        \alpha & \beta & 0 & 0 & 0 \\
        \beta & \alpha'  & \beta & 0 & 0 \\
        0 & \beta & \alpha & \beta & 0 \\
        0 & 0 & \beta & \alpha'  & \beta \\
        0 & 0 & 0 & \beta & \alpha \\
    \end{pmatrix}
\end{equation}
Where $\alpha$ corresponds to the first atom type and $\alpha'$ to the second. Similarly, the value of $\beta$ depends on the interatomic distance.- Therefore, it is possible to express differences in bond lengths between atoms setting two different values of beta $\beta^+$ and $\beta^-$. In the case of pentadiene: 
\begin{equation}
    \hat{H}_{\textrm{Hückel}}  =
    \begin{pmatrix}
        \alpha & \beta^+ & 0 & 0 & 0 \\
        \beta^+ & \alpha & \beta^- & 0 & 0 \\
        0 & \beta^- & \alpha & \beta^+ & 0 \\
        0 & 0 & \beta^+ & \alpha & \beta^- \\
        0 & 0 & 0 & \beta^- & \alpha \\
    \end{pmatrix}
\end{equation}
Where  $\beta^+$ corresponds to a longer bond distance and $\beta^-$ to a shorter one. 

\section{Program review}
The initial intent was to write the program in FORTRAN, but due to problems with my compiler I could not. When calling a matrix diagonalization using llapack, the eigenvalues and eigenvectors were mismatched. Due to this, the program was rewritten in python, where this problem didn't occur. 
\subsection{Code analysis}
The code is divided in two main parts: a class \texttt{Chain} that handles all the properties related to the Hückel Hamiltonian and a function \texttt{run\_calculation} that handles the general flow of a calculation. The advantage of working with a class is that it is possible to create various instances of these chains or rings and compare results and plot them easily and directly. 

\subsubsection{The \texttt{Chain} class}
The \texttt{Chain} class compacts both the calculation and plotting of the eigenvalues and eigenvectors of the system Hamiltonian. The Chain class has the following methods that allow the modification of internal variables:
\begin{lstlisting}
    Chain.close_ring() # sets the internal variable Chain._ring to True
    Chain.open_ring() # sets the internal variable Chain._ring to False
    Chain.set_beta_minus() #  sets the internal variable Chain._beta_minus to False
    Chain.set_alpha_prime() # sets the internal variable Chain._alpha_prime to False
\end{lstlisting} 
These internal variables are accessed when building the Hamiltonian, modifying the different possible cases. 

\paragraph{\texttt{Chain.huckel\_matrix}:}
The Hamiltonian is stored as a property. The function considers different cases in the following way:

\begin{itemize}
    \item Alternating bond distances: in order to simulate different bond distances, the value of the \(\beta\) value is modified in the Hamiltonian. In this case, the value of \(\beta^+\) is set always to \(-1\) while the value of \(\beta^-\) can be set to a different value with \texttt{Chain.set\_beta\_minus}. To take into account this alternation of betas, the \texttt{Chain.huckel\_matrix} modifies the Hamiltonian matrix with:
\begin{lstlisting}[backgroundcolor=\color{LightGray}]
    for i in range(self.n_atoms - 1):
        if i % 2 == 0:
            huckel_matrix[i + 1, i] = huckel_matrix[i, i + 1] = -1
        else:
            huckel_matrix[i + 1, i] = huckel_matrix[i, i + 1] = (
                self._beta_minus
            )
\end{lstlisting} 
    It iterates over the matrix and changes the value of beta when the index is odd. 
    \item Alternating atoms: Similarly to alternating distances, to take into account alternating atoms, it is possible to set a different value to the $\alpha$ value of the diagonal matrix elements with \texttt{Chain.set\_alpha\_prime}. As with the previous property, only the possibility of modifying one of the two $\alpha$ values was considered, and this is reflected in the Hamiltonian matrix with:
\begin{lstlisting}[backgroundcolor=\color{LightGray}]
        for i in range(self.n_atoms):
            if i % 2 == 0:
                huckel_matrix[i, i] = huckel_matrix[i, i] = 0
            else:
                huckel_matrix[i, i] = huckel_matrix[i, i] = self._alpha_prime
\end{lstlisting} 
    
    \item Rings: Rings are built by connecting the first and last atoms of a given chain. This is done explicitly as:
\begin{lstlisting}[backgroundcolor=\color{LightGray}]
    if self._ring:
        huckel_matrix[0, self.n_atoms - 1] = huckel_matrix[self.n_atoms - 1, 0] = -1
\end{lstlisting} 
The condition of the object being a ring is checked and if so the last values of the first row and column are adjusted accordingly. 
        
\end{itemize}

\paragraph{\texttt{Chain.eigen}:} Once the Hamiltonian matrix is built, it can be diagonalized to obtain the eigenvalues and eigenvectors. The Chain.eigen stores as a property the eigenvalues and eigenvectors and calculates it via: 
\begin{lstlisting}[backgroundcolor=\color{LightGray}]
    def eigen(self) -> Tuple[np.ndarray, np.ndarray]:
        self._eigenvalues, self._eigenvectors = np.linalg.eig(
            self.Huckel_matrix
        )
        
        # Sort eigenvalues and eigenvectors
        idx = np.argsort(self._eigenvalues) # returns the array with sorted indexes
        sorted_eigenvalues = self._eigenvalues[idx]
        sorted_eigenvectors = self._eigenvectors[:, idx]

        self._eigenvalues = sorted_eigenvalues
        self._eigenvectors = sorted_eigenvectors

        return sorted_eigenvalues, sorted_eigenvectors
\end{lstlisting} 
The Hückel matrix is diagonalized using the numpy \texttt{linalg.eig} function. Then the pairs of eigenvalues and eigenvectors are sorted to return them in ascending order. 

Finally, the \texttt{Chain} class has two plotting functions:
\begin{lstlisting}[backgroundcolor=\color{LightGray}]
    Chain.plot_eigenvalues() # plots the eigenvalues of the Huckel Hamiltonian
    Chain.plot_eigenvectors() # plots the eigenvectors of the Huckel Hamiltonian
\end{lstlisting} 

\subsubsection{The \texttt{run\_calculation} Function and input handling}
The general scheme of any calculation using the program would be:
\begin{enumerate}
    \item Read the input.
    \item Modify properties accordingly to the input.
    \item Build and diagonalize the Hückel Hamiltonian.
    \item Return the results in a log file and the eigenvalues and eigenvectors in different files for external plotting.
\end{enumerate}

This workflow is managed by the \texttt{run\_calculation} function. The input file is read with:

\begin{lstlisting}[backgroundcolor=\color{LightGray}]
    def run_calculation(input_filename:str, output_filename:str='') -> None:
        content = load_input(input_filename)
    
        if content is None:
            return
    
        keyval_pairs = [ 
            tuple(line.replace('=', ' ').strip().split()[:2])
            for line in content 
            if len(line.replace('=', ' ').strip().split()) >= 2
        ]
    
        cont = dict(keyval_pairs)
    
        if 'n_atoms' not in cont.keys():
            raise ValueError('Definition of number of atoms is mandatory')
    
\end{lstlisting} 

The previous code reads all the content in the input file and generates a dictionary of property-value pairs. This way, it is possible to easily build the \texttt{Chain} object with the correct properties. Also, an error is raised if there is no information of the number of atoms of the system. Then the \texttt{Chain} object is created, and the properties modified with:

\begin{lstlisting}[backgroundcolor=\color{LightGray}]
        c = Chain(n_atoms=int(cont['n_atoms']))

        if 'ring' in cont.keys():
            if cont['ring'].capitalize() == 'True':
                c.close_ring()
    
        if 'alternate_alpha' in cont.keys():
            alt_alpha = float(cont['alternate_alpha'])
            c.set_alpha_prime(alt_alpha)
    
        if 'alternate_beta' in cont.keys():
            alt_beta = float(cont['alternate_beta'])
            c.set_beta_minus(alt_beta)
\end{lstlisting} 

Finally, the eigenvalues and eigenvectors are calculated with:
\begin{lstlisting}[backgroundcolor=\color{LightGray}]
        eigenval, eigenvec = c.eigen    
\end{lstlisting} 

And the results written in the corresponding files with: 
\begin{lstlisting}[backgroundcolor=\color{LightGray}]
        with open(output_filename + '_eigenvalues', 'w') as f:
            for eigen in eigenval:
                f.write(f'{eigen:8.5f}')
        
        with open(output_filename + '_eigenvectors', 'w') as f:
            for vec in eigenvec:
                f.write(' '.join(f'{val:>8.5f}' for val in vec))
\end{lstlisting} 

In order to run the program from the terminal line, the \texttt{sys} module is imported, which allows reading the command line arguments. An error is raised if no input file is provided. 

\subsection{Usage and examples}
The program can be run from the command line by:
\begin{lstlisting}[backgroundcolor=\color{LightGray}]
    python3 huckel.py input_file output_file
\end{lstlisting} 
Where the specification of the output file is optional. A built-in help message is prompted in case of error. The input file should contain a list of keywords:

\begin{lstlisting}[backgroundcolor=\color{LightGray}]
    n_atoms = 100
\end{lstlisting} 
And optional values such as cyclization, distance alternation and atom alternation can be set with:
\begin{lstlisting}[backgroundcolor=\color{LightGray}]
    ring = False
    alternate_alpha = 0.0
    alternate_beta = -1.0
\end{lstlisting} 
The input is case-sensitive, and the program will not run if keywords are not exactly the same. Also, the values True and False must have the first capital letter.

An advantage of working with classes is that the simplicity to calculate and plot different chains and cases, for example to compare the open chain and ring cases it would be:
\begin{lstlisting}[backgroundcolor=\color{LightGray}]
    from huckel import Chain
    a = Chain(10)
    b = Chain(10)
    
    b.close_ring()
    
    a.plot_eigenvalues()
    b.plot_eigenvalues()
\end{lstlisting} 
And the same with the  \texttt{set\_alpha\_prime()} and \texttt{set\_beta\_minus()} methods. All the scripts to obtain the figures in the next section can be found in the \texttt{plots/} directory. 

\section{Results}
Having defined the whole framework in the previous class, it is possible to calculate the eigenvalues and eigenvectors of different systems. The systems that are simulated are the conjugated $\pi$ systems of chains and rings. 
\subsection{Chain vs Ring}
First, the eigenvalues and eigenvectors of both a chain and a ring of 10 atoms were calculated. The results are shown in Figure \ref{fig:chain_ring}. It is possible to appreciate a clear difference between the two cases. The chain system does not present repeated eigenvalues, while the ring presents two-fold degeneracy. This is consistent with MO theory of conjugated rings, where there are degenerated states two by two. 

\begin{figure}[h]
    \centering
    \begin{minipage}{0.47\textwidth}
        \centering
        \includegraphics[width=\textwidth]{Figures/chain_vs_ring.jpg}
        \caption{Comparison of eigenvalues for a chain and a ring of 10 atoms.}
        \label{fig:chain_ring}
    \end{minipage}
    \hfill
    \begin{minipage}{0.47\textwidth}
        \centering
        \includegraphics[width=\textwidth]{Figures/convergence_limit.jpg}
        \caption{Comparison of eigenvalues for a chain and a ring of 100 atoms.}
        \label{fig:chain_vs_ring_infinite}
    \end{minipage}
\end{figure}

Eigenvectors of the first states are shown in Figures \ref{fig:chain_eigenvectors} and \ref{fig:ring_eigenvectors}. There is a significant difference in the eigenvector spectrum. While the wave function of the chain in the ground state is localized in the central atoms in a particle in a box, the wave function of the ring is delocalized over all the atoms. This is a consequence of the periodic boundary conditions of the ring introduced in the Hückel Hamiltonian. Due to the first and last atoms being connected, the ground state wave function will allow free movement of electrons in the wave function, resulting in a delocalized state, again consistent with MO theory of conjugated systems.

The wave function of excited states is also different. In the case of the chain, the wave function behaves as the particle in a box, presenting a number of nodes in the wave function corresponding to the excited state number. On the other hand, the ring presents a slightly different behaviour. Excepting the ground state, the $2n$ and $2n+1$ states present the same eigenfunction, but shifted. Due to the periodic boundary conditions both wave functions behave similarly, which results in the eigenvalue degeneracy. In the case of the ring, the number of nodes of the eigenfunctions will be $2n$ for that pair of eigenfunctions. 

\begin{figure}[h]
    \centering
    \begin{minipage}{0.45\textwidth}
        \centering
        \includegraphics[width=\textwidth]{Figures/chain_eigenvectors.jpg}
        \caption{Eigenvectors for a chain of 10 atoms.}
        \label{fig:chain_eigenvectors}
    \end{minipage}
    \hfill
    \begin{minipage}{0.45\textwidth}
        \centering
        \includegraphics[width=\textwidth]{Figures/ring_eigenvectors.jpg}
        \caption{Eigenvectors for a ring of 10 atoms.}
        \label{fig:ring_eigenvectors}
    \end{minipage}
\end{figure}

Finally, the eigenvalues of the infinite chain/ring limit (considering that a chain/ring of $1000$ atoms converges to said limit) is shown in Figure \ref{fig:chain_vs_ring_infinite}. It is possible to see that, as the ring infinitely large, the eigenvalues of the ring converge to the eigenvalues of the chain. This can be explained considering that an infinitely big ring will behave in the same way as an infinite chain, due to the periodic boundary conditions only being applied at infinity. 

\subsection{Alternating atom distances}
It is possible to simulate the effect of different alternating atoms distances. We refer to the definition of $\beta$ in equation \ref{eq:alpha_beta_equation}. A higher distance between atoms will result in a smaller value of $\beta$, implying that the stabilization due to the neigbouring atoms will be lower. On the other hand a larger value of beta implies proximity and thus higher stabilization. To test the effect of this, figures \ref{fig:chain_alternating_beta_val} and \ref{fig:ring_alternating_beta_val} show the eigenvalues of a chain and a ring with larger and smaller distances between atoms.

\begin{figure}[h]
    \centering
    \begin{minipage}{0.47\textwidth}
        \centering
        \includegraphics[width=\textwidth]{Figures/chain_beta_eigenvaules.jpg}
        \caption{Eigenvalues for a chain with alternating bond distances.}
        \label{fig:chain_alternating_beta_val}
    \end{minipage}
    \hfill
    \begin{minipage}{0.47\textwidth}
        \centering
        \includegraphics[width=\textwidth]{Figures/ring_beta_eigenvalues.jpg}
        \caption{Eigenvalues for a ring with alternating bond distances.}
        \label{fig:ring_alternating_beta_val}
    \end{minipage}
\end{figure}

The first thing that can be noticed is the difference in the eigenvalues. Both the chain and ring with alternating distances present a new band structure with an energy gap between them symmetrical with respect to $0$. This band gap can be associated with the loss of conjugacy generated by the alternating distances. In the case of the linear or ring systems, the whole conjugated $\pi$ system allows free movement of electrons through the whole system, resulting in a gapless conducting system. On the other hand, the differences of distances between neighbours restricts electron movement, resulting in insulation and hence an energy gap between bands. 

On the other hand, the energies of the bands up to the gap differ, with the systems with a smaller beta presenting lower valence band energies. This can be explained due to the Hückel Hamiltonian considering only the stabilization of the $\pi$ system and not the electrostatic repulsion between atoms. Therefore, closer distances will result in stabilizing interactions (lower value of $\beta$) and thus lower ground state energies even though at very small energies the atomic repulsion should increase the energy. 

Also, due to the loss of conjugacy, the two-fold degeneracy of the ring system is lifted, presenting similar eigenvalues for each pair $2n$ and $2n+1$ states, but not exactly the same.

\begin{figure}[h]
    \centering
    \begin{minipage}{0.47\textwidth}
        \centering
        \includegraphics[width=\textwidth]{Figures/ring_beta_eigenvectors.jpg}
        \caption{Ground state eigenvectors of a ring with alternating bond distances.}
        \label{fig:ring_alternating_beta_vec}
    \end{minipage}
    \hfill
    \begin{minipage}{0.47\textwidth}
        \centering
        \includegraphics[width=\textwidth]{Figures/ring_beta_eigenvectors_2.jpg}
        \caption{Higher excited state eigenvectors of a ring with alternating bond distances ($\beta = -1.5$).}
        \label{fig:ring_alternating_beta_vec_2}
    \end{minipage}
\end{figure}

The loss in conjugacy due to alternating atomic distances does not generate a significant change in the ground or excited states of the chain system. On the other hand, the loss of conjugacy in the ring results in localization as it can be appreciated in Figure \ref{fig:ring_alternating_beta_vec}. This localization is extended to the exited states of the system, as it can be seen in Figure \ref{fig:ring_alternating_beta_vec_2}, where the wave functions are not equal and shifted anymore, which explains the loss of the two-fold degeneracy of the ring.  Also, the ring in this condition behaves quite similarly to the chain, where the number of nodes in the wave function corresponds to the excited state.

\subsection{Alternating atom types}
The alternating atom types can be simulated with a variation of the $\alpha$ parameter, as each atom type presents a different value. As the reference value of alpha has been set to $0$ due to \textbf{Equation } \ref{eq:alpha_beta_equation}, \textbf{A lower value of alpha (relative to the reference) implies higher stabilization, while a higher one a more unstable interaction }. The alternation of $\alpha$ values will result again in the loss of conjugacy. The eigenvalues of the chains and rings are plotted in Figures \ref{fig:chain_alternating_alpha_val} and \ref{fig:ring_alternating_alpha_val}, while eigenvectors in Figures \ref{fig:chain_alternating_alpha_vec} and \ref{fig:ring_alternating_alpha_vec}.

\begin{figure}[h]
    \centering
    \begin{minipage}{0.47\textwidth}
        \centering
        \includegraphics[width=\textwidth]{Figures/alpha_in_chain.jpg}
        \caption{Eigenvalues for a chain with alternating atom types.}
        \label{fig:chain_alternating_alpha_val}
    \end{minipage}
    \hfill
    \begin{minipage}{0.47\textwidth}
        \centering
        \includegraphics[width=\textwidth]{Figures/alpha_in_ring.jpg}
        \caption{Eigenvalues for a ring with alternating atom types.}
        \label{fig:ring_alternating_alpha_val}
    \end{minipage}
\end{figure}

As with the alternating distances, the alternating atom types generate a loss conjugacy and thus a band gap in the energy spectrum. It can be appreciated that in this case the band gap is not the same as in the alternating distance cases, where there gap starts at $E = 0$. 

% due to somethoing something diagonal and non-diagonal terms. diagonal not only generates loss of conjugacy but also shfts the whole spectrum. 

\begin{figure}[h]
    \centering
    \begin{minipage}{0.47\textwidth}
        \centering
        \includegraphics[width=\textwidth]{Figures/chain_alpha_eigenvectors.jpg}
        \caption{Ground state eigenvectors of a chain with alternating atom types.}
        \label{fig:chain_alternating_alpha_vec}
    \end{minipage}
    \hfill
    \begin{minipage}{0.47\textwidth}
        \centering
        \includegraphics[width=\textwidth]{Figures/ring_alpha_eigenvectors.jpg}
        \caption{Ground state eigenvectors of a ring with alternating atom types.}
        \label{fig:ring_alternating_alpha_vec}
    \end{minipage}
\end{figure}

% \printbibliography

\end{document}
